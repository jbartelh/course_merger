%% Erläuterungen zu den Befehlen erfolgen unter
%% diesem Beispiel.

\documentclass{scrartcl}

\usepackage[utf8]{inputenc}
\usepackage[T1]{fontenc}
\usepackage{lmodern}
\usepackage[ngerman]{babel}
\usepackage{amsmath}
\usepackage{listings}
\usepackage{hyperref}
\usepackage{nameref}
\lstset{
language=bash,
frame=single,
basicstyle=\footnotesize
}

\usepackage{parskip}
%\setlength{\parindent}{0pt}

\title{Open edX Dokumentation - Administration}
\author{Joel Bartelheimer}

\begin{document}

\maketitle

\tableofcontents

\section{Administration - Backend}
\label{sec:admin}

Die edx-platform ist die Kernkomponente des Open-edx Systems. Sie enthält das learning management system(LMS) und das Studio(CMS) zum organisieren von Kursinhalten. Programmiert wurde die edx-platform fast ausschließlich in Python mit der verwendung des \href{https://www.djangoproject.com/}{Django} Webframeworks.

Das Django Framework kommt standardmäßig mit einer Reihe an Konfigurationen, wie dem Django Adminbereich, welche auch bei Open-edX verwendet werden können. In dem Django Adminbereich können beispielsweise Benutzer mit den dazugehörigen Rollen gepflegt werden. Für die Benutzerverwaltung mithilfe des Django Adminbereiches wird an dieser Stelle auf die Dokumentation der Firma Abstract-Technology GmbH verwiesen(\textit{[ABSTRACT]-HS-Osnabrueck-UserManagement.V1.1.pdf})

Eine weiteres Wichtiges Werkzeuge neben dem Django Adminbereich ist die Django Shell, welche im folgendem Kapitel erklärt wird

\subsection{SSH Backend-Login}
\label{sec:ssh}

Mit SSH wird auf das Backend von Open-edX zugegriffen. Eine IP-Adresse, Port und Benutzer für das jeweilige System muss dafür vorhanden sein.

\begin{lstlisting}
$ ssh <User>@<edx-Ip-Adresse> -p <Port>
\end{lstlisting}

Die Verbindung zur Open-edX Testumgebung \textit{edx-devel} mit dem User \textit{edx-admin} sieht wie folgt aus:

\begin{lstlisting}
$ ssh edx-admin@131.173.108.215 -p 14301
\end{lstlisting}



\subsection{Django Shell commands}

Als erstes sei an dieser Stelle auf den jeweiligen Wiki-Bereich der \textit{edx-platform} auf Github verweisen(\href{https://github.com/edx/edx-platform/wiki/Shell-commands}{Wiki Shell commands}), da dort ebenfalls einiges Dokumentiert ist.

Für diverse administrative Aufgaben am Open-edX System wird die durch Django gelieferte \lstinline|manage.py| verwendet. Der syntax zur verwendung sieht wie folgt aus:

\begin{lstlisting}
$ path/to/edx-platform/manage.py  [lms|cms]  --settings=aws <cmd>
\end{lstlisting}

Zum Ausführen muss der Pfad zur Open-edX Installation angegeben werden (\textit{path/to/edx-platform}), zum Ausführen werden spezielle User-Rechte benötigt und zusätzlich muss eine spezielle Python Umgebung für Open-edX verwendet werden. Mit \textit{lms} oder \textit{cms} wird entweder das \textit{learning management System} oder die Kursverwaltung \textit{Studio} angesprochen. Mit \textit{<cmd>} wird der auszuführende Befehl angegeben.

\begin{lstlisting}[caption={help command}, label=lst:django_help]
$ cd /edx/app/edxapp/edx-platform/
$ sudo -u edxapp /edx/bin/python.edxapp manage.py cms --settings=aws help
\end{lstlisting}

In Listing~\ref{lst:django_help} wird als erstes in das Verzeichnies der Open-edX Installation gewechselt. Der nachfolgende Befehl wird als \textit{edxapp}-User ausgeführt, wobei \textit{/edx/bin/python.edxapp} ein Verweis auf die spezielle Python Umgebung darstellt und der \lstinline|help|-Befehl alle verfügbaren Befehle für das Studio(cms) auflistet. Die Ausgabe des |help|-Befehl ist in Listing~\ref{lst:output_help_cms} dargestellt.

Die verfügbaren Befehle für das \textit{lms} unterscheiden sich nur geringfügig von denen des Studios, da die meisten Befehle vom Django stadardmäßig bereigestellt werden. Eine kurze Auflistung der speziellen \textit{lms}-Befehle ist in Listing~\ref{lst:output_help_lms} zu sehen.

\begin{lstlisting}[caption={Ausgabe von help (auflisten aller Befehle) für das Studio}, label=lst:output_help_cms, breaklines]
Usage: manage.py subcommand [options] [args]

Options:
  -v VERBOSITY, --verbosity=VERBOSITY
                        Verbosity level; 0=minimal output, 1=normal output,
                        2=verbose output, 3=very verbose output
  --settings=SETTINGS   The Python path to a settings module, e.g.
                        "myproject.settings.main". If this isn't provided, the
                        DJANGO_SETTINGS_MODULE environment variable will be
                        used.
  --pythonpath=PYTHONPATH
                        A directory to add to the Python path, e.g.
                        "/home/djangoprojects/myproject".
  --traceback           Print traceback on exception
  --version             show program's version number and exit
  -h, --help            show this help message and exit

Type 'manage.py help <subcommand>' for help on a specific subcommand.

Available subcommands:

[auth]
    changepassword
    createsuperuser

[contentstore]
    check_course
    cleanup_assets
    clone_course
    create_course
    delete_course
    delete_orphans
    edit_course_tabs
    empty_asset_trashcan
    export
    export_all_courses
    export_convert_format
    fix_not_found
    git_export
    import
    migrate_to_split
    populate_creators
    prompt
    restore_asset_from_trashcan
    utils
    xlint

[django]
    cleanup
    compilemessages
    createcachetable
    dbshell
    diffsettings
    dumpdata
    flush
    inspectdb
    loaddata
    makemessages
    reset
    runfcgi
    shell
    sql
    sqlall
    sqlclear
    sqlcustom
    sqlflush
    sqlindexes
    sqlinitialdata
    sqlreset
    sqlsequencereset
    startapp
    startproject
    validate

[django_nose]
    test

[django_openid_auth]
    openid_cleanup

[djcelery]
    camqadm
    celery
    celerybeat
    celerycam
    celeryctl
    celeryd
    celeryd_detach
    celeryd_multi
    celeryev
    celerymon
    djcelerymon

[edxmako]
    preprocess_assets

[openassessment]
    create_oa_submissions
    simulate_ai_grading_error
    upload_oa_data

[require]
    require_init

[south]
    convert_to_south
    datamigration
    graphmigrations
    migrate
    migrationcheck
    schemamigration
    startmigration
    syncdb
    testserver

[static_replace]
    clear_collectstatic_cache

[staticfiles]
    collectstatic
    findstatic
    runserver

[student]
    6002exportusers
    6002importusers
    add_to_group
    anonymized_id_mapping
    assigngroups
    cert_restriction
    change_enrollment
    create_random_users
    create_user
    emaillist
    get_grades
    set_staff
    sync_user_info
    transfer_students
    userinfo

[user_api]
    email_opt_in_list

\end{lstlisting}

\begin{lstlisting}[caption={Ausgabe von help (auflisten aller Befehle) für das lms}, label=lst:output_help_lms, breaklines]
...
[certificates]
    cert_whitelist
    fix_ungraded_certs
    gen_cert_report
    regenerate_user
    ungenerated_certs

[courseware]
    clean_history
    clean_xml
    dump_course_ids
    dump_course_structure
    export_course
    import
    metadata_to_json
    regrade_partial
    remove_input_state
...
\end{lstlisting}

\subsubsection{Courseware - MongoDB}
Die Kursinhalte sind bei Open-edX in einer eigenen noSQL-Datenbank gespeichert. Da die meisten Befehle der Django-Shell nur Datenbankoperation sind, könnte man prinziepiell auch direkt auf der Datenbank operieren statt die Befehle zu verwenden. Um keine Kurse zu zerstören sollten Schreiboperationen mit höchster Vorsicht verwendet werden.

\paragraph{MongoDB zugriff über SSH-Sitzung}

Die \href{https://docs.mongodb.com/getting-started/shell/client/}{MongoDB Shell} wird mit dem Befehl \lstinline|mongo| gestartet, siehe Listing~\ref{lst:mongo}.

\begin{lstlisting}[caption={MongoDB shell}, label=lst:mongo]
edx-admin@edx-devel:~$ mongo
MongoDB shell version: 2.6.11
connecting to: test
> 
\end{lstlisting}

Die für Open-edX interessante Datenbank ist die \textit{edxapp}, mit \lstinline|use edxapp| wird diese verwendet. Listing~\ref{lst:mongo_list_courses} zeigt wie über die MongoDB-Shell alle existierenden Kurse aufgelistet werden können.

\begin{lstlisting}[caption={Auflisten aller Kurse}, label=lst:mongo_list_courses]
> use edxapp
db.modulestore.find( { "_id.category" : "course" }, {'name':'1'} )
\end{lstlisting}

\paragraph{MongoDB zugriff über GUI}

Statt über die Konsole kann auch eine GUI-Anwendung für den Datenbankzugriff verwendet werden, z.B. Robomongo. Die Verbindungsparameter für den Zugriff mit Robomongo:

\begin{itemize}
\item Connection Method: Standard TCP/IP over SSH
\item SSH Hostname: siehe Kapitel~\ref{sec:ssh}
\item SSH Username: siehe Kapitel~\ref{sec:ssh}
\item Address: 127.0.0.1
\item Port: 27017
\end{itemize}

\subsubsection{MySQL-Datenbank}

Für alle weiteren Daten, neben den Kursinhalten, verwendet Open-edX eine MySQL-Datenbank. Auf diese kann ebenfalls mittels Shell oder GUI zugegriffen werden und Schreiboperationen sind auch hier kritisch.

Ein Beispiel für die verwendung der MySQL-Shell ist im Kapitel~\ref{sec:uc_list_users} Abschnitt \nameref{p:uc_mysql} dargestellt.

Verbindungsparameter zur MySQL-Datenbank mit der MySQL Workbench:

\begin{itemize}
\item Connection Method: Standard TCP/IP over SSH
\item SSH Hostname: siehe Kapitel~\ref{sec:ssh}
\item SSH Username: siehe Kapitel~\ref{sec:ssh}
\item MySQL Hostname: 127.0.0.1
\item MySQL Server Port: 3306
\item Username: root (no password)
\item  Default Schema: edxapp
\end{itemize}

\subsection{Anwendungsfall: Kurs löschen}

Die Django-Shell für das Studio bietet zum Löschen eines Kurses den Befehl \lstinline|delete_course| an. Der Befehl bietet zwei Parameter an, die Kurs-Id und den optionalen Parameter \lstinline|commit|. Mit dem \textit{lms}-Befehl \lstinline|dump_course_ids| kann eine Liste aller Kurs-Ids ausgegeben werden.

\begin{lstlisting}[caption={Syntax \lstinline|delete_course|}, label=lst:delete_course_syntax, breaklines]
manage.py cms --settings=aws delete_course <course_id> [commit]
\end{lstlisting}

Durch Angabe der jeweiligen Kurs-Id und ohne Commit kann vorab geprüft werden ob Fehler auftreten. Wenn alles in Ordnung ist, kann mit \lstinline|commit| und zwei weiteren Bestätigungen der Kurs gelöscht werden. In Listing~\ref{lst:delete_course} ist dieser Ablauf einemal dargestellt.

\begin{lstlisting}[caption={Ausgabe nach \lstinline|delete_course| erst ohne und dann mit commit}, label=lst:delete_course, breaklines]
edx-admin@edx-devel:/edx/app/edxapp/edx-platform$ sudo -u edxapp /edx/bin/python.edxapp manage.py cms --settings=aws delete_course qqq/qqq/qqq
2016-09-23 07:10:01,197 INFO 30743 [dd.dogapi] dog_stats_api.py:66 - Initializing dog api to use statsd: localhost, 8125

edx-admin@edx-devel:/edx/app/edxapp/edx-platform$ sudo -u edxapp /edx/bin/python.edxapp manage.py cms --settings=aws delete_course qqq/qqq/qqq commit
2016-09-23 07:10:28,880 INFO 30848 [dd.dogapi] dog_stats_api.py:66 - Initializing dog api to use statsd: localhost, 8125
Actually going to delete the course from DB....
Deleting course qqq/qqq/qqq. Confirm? [y/N] y
Are you sure. This action cannot be undone! [y/N] y
removing User permissions from course....
edx-admin@edx-devel:/edx/app/edxapp/edx-platform$
\end{lstlisting}

\subsection{Anwendungsfall: Benutzer auflisten}
\label{sec:uc_list_users}

Hier werden einmal zwei verschiedene Varianten dargestellt, welche alle im System angelegten Benutzer auflistet. Benutzer mit dem Status \lstinline|is_superuser = true| können beim Login in den Django Adminbereich (siehe Kapitel~\ref{sec:admin}, Dokumantation der Fa. Abstract-Technology) verwendet werden.

\paragraph{Django-Shell}

In der Django-Shell wird Python-Code verwendet, welcher über den integrierten ORM(object-relational mapping) auf die Datenbank zugreift. Die zurückgegebenen Objekte beinhalten nicht nur den Namen, sondern auch alle anderen Informationen. Mit \lstinline|user[0].email| wird beispielsweise die E-Mail von dem Benutzer \textit{honor} ausgegeben. Da es sich um Python-Code handelt kann noch weitere Logik implementiert oder ganze Python-Scripte geschrieben werden.

\begin{lstlisting}[caption={Auflisten aller Benutzer(Django)}, label=lst:django_list_users, breaklines]
$ sudo -u edxapp /edx/bin/python.edxapp manage.py cms --settings=aws shell
2016-09-23 05:31:56,314 INFO 20588 [dd.dogapi] dog_stats_api.py:66 - Initializing dog api to use statsd: localhost, 8125
Python 2.7.3 (default, Jun 22 2015, 19:33:41) 
[GCC 4.6.3] on linux2
Type "help", "copyright", "credits" or "license" for more information.
(InteractiveConsole)
>>> from django.contrib.auth.models import User
>>> user = User.objects.all()
>>> user
[<User: honor>, <User: audit>, <User: verified>, <User: staff>, <User: fruend>, <User: user>]

\end{lstlisting}

\paragraph{MySQL-Shell}
\label{p:uc_mysql}

Das Listing~\ref{lst:mysql_list_users} zeigt einen Zugriff auf die MySQL-Datenbank mittels der MySQL-Shell. Mit dem Benutzer \textit{root} wird sich mit der lokalen MySQL-Instanz verbunden und mit dem \lstinline|use edxapp| Befehl auf das passende DB-Schema gewechselt. Anshcließend folgt ein einfacher SQL-Befehl der alle Einträge aus der Tabelle \lstinline|auth_user| ausgiebt.

\begin{lstlisting}[caption={Auflisten aller Benutzer(MySQL)}, label=lst:mysql_list_users, breaklines]
$ mysql --user=root
Welcome to the MySQL monitor.  Commands end with ; or \g.
Your MySQL connection id is 1373249
Server version: 5.6.24-2+deb.sury.org~precise+2 (Ubuntu)

Copyright (c) 2000, 2015, Oracle and/or its affiliates. All rights reserved.

Oracle is a registered trademark of Oracle Corporation and/or its
affiliates. Other names may be trademarks of their respective
owners.

Type 'help;' or '\h' for help. Type '\c' to clear the current input statement.

mysql> use edxapp
Reading table information for completion of table and column names
You can turn off this feature to get a quicker startup with -A

Database changed
mysql> SELECT * FROM edxapp.auth_user;

\end{lstlisting}

\section{Link Sammlung}

\href{http://docs.edx.org/}{Documentation for edx.org and the Open edX Community}

\begin{description}

\item[Installing, Configuring, and Running the Open edX Platform]\hfill \\
Anleitungen zum Hosten einer edX-Installation für den produktiven oder entwicklungs Einsatz. Changelog der verschiedenen Releases. Konfigurieren der EdX-Plattform, wie z.B. das Installieren eines XBlock.
 \href{http://edx.readthedocs.io/projects/edx-installing-configuring-and-running/en/latest/}{Installing, Configuring, and Running the Open edX Platform}

\item[Open edX Installation Options (Atlassian-Wiki)]\hfill \\
Anleitungen zum Hosten einer edX-Installation für den produktiven oder entwicklungs Einsatz
 \href{https://openedx.atlassian.net/wiki/display/OpenOPS/Open+edX+Installation+Options}{Open edX Installation Options}

\item[EdX Open Learning XML Guide]\hfill \\
Dokumentation für das XML-basierte Format für edX-Kurse. Aufbau, Struktur und Beispiele für das Erstellen und Bearbeiten von Kursen mit dem XML-Format.
 \href{http://edx.readthedocs.io/projects/edx-open-learning-xml/en/latest/}{EdX Open Learning XML Guide}

\item[edX Github Repositories]\hfill \\
Alle Git-Repositories von edX und  deren Dokumentationen.
 \href{https://github.com/edx}{EdX auf Github}

Interesannte Repositories:

\begin{itemize}
\item \href{https://github.com/edx/edx-platform/wiki}{edx-platform} Kernkomponente mit Wiki in dem unteranderem Häufige Probleme und technische Sachen wie die Django Shell-Commands erklärt werden.
\item \href{https://github.com/edx/edx-tools/wiki}{edx-tools} Sammlung von hilfreichen Tools und Python-Scripts wie z.B. dem \textit{Clean Studio XML}.
\end{itemize}

\end{description}

\end{document}
